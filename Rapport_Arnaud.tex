\documentclass[]{article}
\usepackage{lmodern}
\usepackage{amssymb,amsmath}
\usepackage{ifxetex,ifluatex}
\usepackage{fixltx2e} % provides \textsubscript
\ifnum 0\ifxetex 1\fi\ifluatex 1\fi=0 % if pdftex
  \usepackage[T1]{fontenc}
  \usepackage[utf8]{inputenc}
\else % if luatex or xelatex
  \ifxetex
    \usepackage{mathspec}
  \else
    \usepackage{fontspec}
  \fi
  \defaultfontfeatures{Ligatures=TeX,Scale=MatchLowercase}
\fi
% use upquote if available, for straight quotes in verbatim environments
\IfFileExists{upquote.sty}{\usepackage{upquote}}{}
% use microtype if available
\IfFileExists{microtype.sty}{%
\usepackage{microtype}
\UseMicrotypeSet[protrusion]{basicmath} % disable protrusion for tt fonts
}{}
\usepackage[margin=1in]{geometry}
\usepackage{hyperref}
\hypersetup{unicode=true,
            pdftitle={Probabilités Appliquées - Projet simulation de file d'attente},
            pdfauthor={Arnaud PIERRET},
            pdfborder={0 0 0},
            breaklinks=true}
\urlstyle{same}  % don't use monospace font for urls
\usepackage{graphicx,grffile}
\makeatletter
\def\maxwidth{\ifdim\Gin@nat@width>\linewidth\linewidth\else\Gin@nat@width\fi}
\def\maxheight{\ifdim\Gin@nat@height>\textheight\textheight\else\Gin@nat@height\fi}
\makeatother
% Scale images if necessary, so that they will not overflow the page
% margins by default, and it is still possible to overwrite the defaults
% using explicit options in \includegraphics[width, height, ...]{}
\setkeys{Gin}{width=\maxwidth,height=\maxheight,keepaspectratio}
\IfFileExists{parskip.sty}{%
\usepackage{parskip}
}{% else
\setlength{\parindent}{0pt}
\setlength{\parskip}{6pt plus 2pt minus 1pt}
}
\setlength{\emergencystretch}{3em}  % prevent overfull lines
\providecommand{\tightlist}{%
  \setlength{\itemsep}{0pt}\setlength{\parskip}{0pt}}
\setcounter{secnumdepth}{0}
% Redefines (sub)paragraphs to behave more like sections
\ifx\paragraph\undefined\else
\let\oldparagraph\paragraph
\renewcommand{\paragraph}[1]{\oldparagraph{#1}\mbox{}}
\fi
\ifx\subparagraph\undefined\else
\let\oldsubparagraph\subparagraph
\renewcommand{\subparagraph}[1]{\oldsubparagraph{#1}\mbox{}}
\fi

%%% Use protect on footnotes to avoid problems with footnotes in titles
\let\rmarkdownfootnote\footnote%
\def\footnote{\protect\rmarkdownfootnote}

%%% Change title format to be more compact
\usepackage{titling}

% Create subtitle command for use in maketitle
\newcommand{\subtitle}[1]{
  \posttitle{
    \begin{center}\large#1\end{center}
    }
}

\setlength{\droptitle}{-2em}
  \title{Probabilités Appliquées - Projet simulation de file d'attente}
  \pretitle{\vspace{\droptitle}\centering\huge}
  \posttitle{\par}
  \author{Arnaud PIERRET}
  \preauthor{\centering\large\emph}
  \postauthor{\par}
  \predate{\centering\large\emph}
  \postdate{\par}
  \date{1 Juin 2018}


\begin{document}
\maketitle

\subsection{Introduction}\label{introduction}

Ce projet fut réalisé à l'aide du langage R. Le problème présenté est de
simuler une file d'attente sur un serveur.

\subsection{Arrivé et traitement des
requêtes}\label{arrive-et-traitement-des-requetes}

Les requêtes arrivent aux temps A1,A2,..,An en suivant la une loi
exponentielle de paramètre \$lambda. Afin de gérer les niveaux de
priorités (similaire à ce que ferai un système de QoS), les requêtes
entrantes sont stockées dans une des trois sous-file de priorité avec le
choix de la file réalisé grâce à une variable aléatoire suivant une loi
uniforme. La file d'attente a une taille N qui limite le nombre de
requêtes dans le système. Si une requête arrive alors que la file est
pleine, cette requête est rejetée.

Les requêtes sont traitées aux temps D1,D2,..,Dn en suivant également
une loi exponentielle mais cette fois ci de paramètre \$(1/mu).
Lorsqu'un temps de traitement arrive, une requête (située dans une des
files de priorité) est retirée de la simulation. La gestion des
priorités fut réalisée de manière assez naïve comparé à la réalité. En
effet, dans notre simulation, tant que des paquets tagués P1 sont en
attentes ils sont traités, puis les P2, puis les P3. Dans la vraie vie,
des systèmes de traitement de queue tels que le Round Robin and Weighted
Fair Queuing (WFQ) sont plus utilisés. Pour autant, dans le contexte de
notre simulation, le Priority Queuing (PQ) donne déjà des résultats
probants. Pour ce projet, j'ai tenté de mettre en place un WFQ en
limitant à 3 traitements successifs pour chaque priorité. La différence
avec le PQ était ici négligeable.

\subsection{Influence des paramètres lambda et
mu}\label{influence-des-parametres-lambda-et-mu}

Tout d'abord, nous allons effectuer des expériences nous montrant
notamment les effets des variations de lambda et mu sur le système. Nous
allons étudier pour le moment 3 cas, lambdamu Pour ce faire, lançons une
simulation avec un lambda allant de 1 à 10 et étudions la perte des
paquets. Les autres paramètres pour cette expérience seront : - mu = 1 -
N = 10 - p1 = 0.5 - p2 = 0.3 - p3 = 0.2 - durée de l'expérience :
10\^{}4

\includegraphics{Rapport_Arnaud_files/figure-latex/unnamed-chunk-1-1.pdf}

En dessous du rapport \(\lambda/\mu\) le nombre de paquets perdus est
négligeable, il tend vers 0. Ce qui est logique puisque si le traitement
se fait plus rapidement que les arrivés et avec une file de 10 éléments,
il y a peu de risque de perdre des paquets. En revanche, on peut
observer qu'au delà d'un ratio \(\lambda/\mu\) supérieur à 4, le nombre
de paquets perdus croit rapidement jusqu'à dépasser 12000 pertes pour un
ratio de 10

\subsection{Perte de paquets}\label{perte-de-paquets}

Le pourcentage de paquets perdus théorique est donné par cette formule :
\[P(X_t=N)=\frac{1-\rho}{1-\rho^{N+1}}\rho^N\]

Pour trouver une valeur empirique à cette probabilité nous pouvons
tracer le pourcentage de paquets perdus sur le total de paquets arrivés
pour différentes valeurs de \(\lambda\).

\includegraphics{Rapport_Arnaud_files/figure-latex/unnamed-chunk-2-1.pdf}

Le taux de perte est donc élevé puisqu'il monte au delà de 40\% pour un
ratio \(\lambda/\mu\)=10.

Afin de résoudre ce problème on peut soit augmenter la taille du buffer
d'attente ou encore ajouter des serveurs. C'est ce que nous allons voir.

\subsection{Effets de l'augmentation du nombre de
serveurs}\label{effets-de-laugmentation-du-nombre-de-serveurs}

Pour simuler une augmentation du nombre de serveur, nous avons choisis
d'opter pour une boucle au sein de notre partie traitement de requêtes
Ainsi, à chaque fois qu'une période de traitement arrive, on ne traite
pas 1 mais N requêtes. Par convention, nous avons également choisi de
multiplier la taille de la file passé en paramètre par N. Une autre
méthode qui aurait pu être mis en place aurait été de multiplier le
paramètre \(\mu\) par N pour diviser le temps avant le prochain
traitement. Sur le graphe ci-dessous, vous pouvez voir les effets de
l'augmentation du nombre de serveurs.

\includegraphics{Rapport_Arnaud_files/figure-latex/unnamed-chunk-3-1.pdf}

Pour une queue de 10 éléments et un ratio de \(\lambda/\mu\) de 12, il
suffit de 4 serveurs pour arriver à un nombre nul ou quasi nul de
paquets perdus.


\end{document}
